\documentclass[11pt]{memoir}

\usepackage{classpaper}

\title{Whether the ultimate disposition to grace proceeds effectively from habitual grace?}
\author{John of St. Thomas}
\date{}

\begin{document}

\maketitle

\noindent \emph{Notes from \emph{Cursus Theologiae}, disp. 28, a. 2. This is from John of St. Thomas' discussion of justification
in his treatise on grace. Here he is commenting on \emph{Summa Theologiae} IaIIae, q. 113, a. 8}

\section*{Statement of the Problem}

In the last article, John of St. Thomas inquired into the remote and proximate dispositions for justification. First, he
clarifies the notion of a disposition. John points out that dispositions ought to have some causality, and hence some
connection with respect of the form toward which they dispose. Specifically, dispositions
are in the genus of material cause. But not just any material cause is a disposition, for a subject related
indifferently to the reception of the form is not a disposition. Rather, a disposition is a material cause as
determining and proportioning the very subject with respect to the form.\footnote{d. 28, a. 1, n. 2} \\

\noindent Generally speaking, dispositions come in two flavors: \emph{remote} and \emph{proximate}. A remote disposition is what
pertains to the beginning of the introduction of some form. A proximate disposition is that by which the introduction to
the form is finally accomplished [\emph{consummatur}]. For this reason, a proximate disposition is also called an
ultimate disposition.\footnote{d. 28, a. 1, n. 2} \\

\noindent John concludes that faith is the first remote disposition for justification. It is a necessary beginning for
justification, but more is required for justification to come about.\footnote{d. 28, a. 1, n. 3} The fundamental reason
why Protestants err in this area is that they see justification as extrinsic, where sin is only covered and not imputed.
The Catholic faith, on the other hand, teaches that justification is the true, intrinsic renewal of the inner
man.\footnote{d. 28, a.1, n.4} \\

\noindent With respect of the will, what acts are required? Many are helpful, but not all are \emph{per se}
necessary.\footnote{d. 28, a. 1, n. 9} The ones
that are \emph{per se} necessary are hope of pardon, \emph{dilectio}, and penance.\footnote{d. 28, a. 1, n. 10} Some
have held that all of these acts must be explicitly and disctincly elicited. But others have said more truly that an
efficacious act of \emph{dilectio} virtually contains the other two.\footnote{d. 28, a. 1, n. 11} \\

\noindent The true and proper ultimate dispositions for justification, then, are \emph{dilectio} and perfect
contrition.\footnote{d. 28, a. 1, n. 12} But, although contrition and \emph{dilectio} are perfect insofar as they are
dispositions, nevertheless, they still need a superadded favor and mercy of God. It is habitual grace that formally and
intrinsically justifies. Contrition and \emph{dilectio} dispositively and effectively turn a person to God, but they do
not turn God toward the person. This is an effect of the will of God mercifully remitting sin (which must produce a real
change in the person, but that is another topic). The dispositions of contrition and \emph{dilectio} turn us to God not
as dispositions which actively introduce grace, but which passively dispose us to receive from God, as a body must first
be turned to a source of light in order to then be enlightened by it.\footnote{d. 28, a. 2, n. 12} \\

\noindent But all of this immediately gives rise to a question. In any change, the ultimate disposition for the reception of a
form is inseperably and \emph{per se} conjoined to the form that is to be introduced. Does the ultimate disposition
proceed from the form in the genus of efficient cause? Specifically in our situation: do contrition and \emph{dilectio}
proceed effectively from habitual grace? \\

\noindent There is a lot of debate around these points, but, in general, there are two basic teachings: (1) The ultimate
disposition proceeds effectively from habitiual grace; and (2) the ultimate disposition proceeds effectively from actual
grace, and not habitual grace.

\section*{The First Position: \emph{Dilectio} and Contrition Proceed from Habitual Grace}

    \subsection*{Philosophical Foundation}

    Although the disposition is the cause of the form in the genus of quasi-material cause, nevertheless, in the other
    genera of causes, the disposition can be the effect of the form. This is because causes are causes to each other and
    are related as prior and posterior according to diverse dependencies and the urgency of nature. Therefore, although
    the disposition precedes the form in the genus of material cause, nevertheless, it follows the form in the genus of
    efficient cause, and is cause by and depends on it.\footnote{d. 28, a. 2, n. 3} \\
    
    \noindent In fact, St. Thomas says in the passage that we are
    considering\footnote{IaIIae, q. 113, a. 8} that the infusion of habitual grace precedes the motion of the free will.
    This means that the infused grace has the \emph{ratio} of a cause with respect to the motion of the free
    will.\footnote{d. 28, a. 2, n. 4}

    \subsection*{First Theological Foundation}

    \noindent Contrition, as it is an ultimate disposition, merits glory, but does not merit grace.\footnote{Ia, q. 112, a. 2, ad
    1} Therefore it ought to proceed from habitual grace. \\

    \noindent \textbf{Proof.} The motion of contrition cannot merit glory unless it proceeds from grace. This is because if it
    does not proceed from a graced subject and the love of could, it would in no way merit before God, because merit is
    not only with respect to acts, but to persons. Nor would it suffice for contrition to have grace for its term in
    order for contrition to merit glory, because in this case grace would no longer be the principle of
    merit.\footnote{d. 28, a. 2, n. 5} \\

    \noindent \textbf{Confirmation.} If it were the case that contrition and \emph{dilectio} did not proceed from habitual grace,
    then they could be unformed, as acts of faith and hope.\footnote{d. 28, a. 2, n. 6} 

    \subsection*{Second Theological Foundation}

    \begin{enumerate}
        
        \item Christ the Lord has been justified with respect to accidental sanctification through a proper disposition
            as an adult.

        \item That act by which he disposed himself proceeds efficiently from habitual grace.

        \item Therefore, it can also proceed [this way] in us.\footnote{d. 28, a. 2, n. 7}

    \end{enumerate}

    \noindent \textbf{Proof of (1)} This is the common teaching of the theologians and St. Thomas.\footnote{cf. IIIa, q. 7 and
    34.} \\

    \noindent \textbf{Proof of (2)} If that disposition did not proceed from habitual grace, then the disposition would merit the
    habitual grace, and consequently glory, which is grace consummated. But the common teaching of theologians (as well
    as St. Thomas) denies this.\footnote{ibid.} \\

    \noindent This can be shown by an argument. Suppose that the disposition did precede habitual grace. Even then, it
    would be a human act that proceeded from a divine person united to the humanity. Therefore, the dispositive action
    was sanctified and proceeded from a subject sanctified with a personal sanctity, therefore it is in just this way
    that it merited the grace to which it disposed. Note, however, that this argument does not apply to the angels, who
    were justified through a proper act, because altho that act did not proceed from habitual grace but a personal
    elevation through actual grace, nevfertheless, because the person himself was not sanctified prior to habitual
    grace, the act of such a person was not precisely as such meritorious.\footnote{ibid.} \\

    \noindent [NB: I don't exactly understand what John of St. Thomas is getting at here. These two paragraphs seem to be saying
    different things.]\footnote{ibid.} \\
    
    \noindent \textbf{Proof of (3)} If that disposition proceeds from grace in one subject, then it is not unfitting that it so
    proceeds in others.\footnote{ibid.} 


\section*{The Second Position: \emph{Dilectio} and Contrition Proceed from Actual Grace}

    \subsection*{Philosophical Foundation}

    Disposition to a form precedes the form and is its cause in the order of disposition. Therefore, it cannot be the
    effect of the form intruduced in the genus of efficient cause. \\

    \noindent \textbf{Argument.} In order for the disposition to be the effect of the form in the genus of efficient cause, it
    ought to suppose that the form exists in act and not outside of, but inside of, the subject. Thus the disposition
    proceeds from a form already introduced and received into the subject.\footnote{d. 28, a. 2, n. 2} 

    \subsection*{Theological Foundation}

    \noindent Contrition and love of God, no matter how perfect, need a new grace and favor of God in order for sanctifying
    habitual grace to be infused. Therefore they cannot be the effects of habtual grace itself in the genus of efficient
    cause.\footnote{d. 28, a. 2, n. 8} \\

    \noindent \textbf{Proof of Antecedent.} Immediate from the last article. \\

    \noindent \textbf{Proof of Consequent.} If contrition, insofar as it is an ultimate disposition, is an effect of
    habitual grace, then it does not require another favor and will of God beyond that which infuses the habitual grace.
    Again, if the grace itself is infused by God sufficiently for the remission
    of sin and the turning of God to the person, then contrition is an effect of the grace derived from such a will, and
    thus does not need another gratuitous will from God for remitting sin and turning God to the person.\footnote{ibid.}
    (Note that this is a proof of the contrapositive). \\

    \noindent \textbf{First Confirmation.} Those dispositions by which we are disposed for justification are not proper
    passions flowing from grace. Rather, they are elicited acts by which we are turned to God so that we may be
    mercifully illuminated by Him. Thus, those dispositions do not introduce necessity, as do the natural dispositions
    for a form, as St. Thomas says.\footnote{\emph{De veritate}, q. 28, a. 2, ad 8} Consequently, such a disposition is
    not a proper passion proceeding from grace through a necessary flowing. Hence, if the disposition does proceed
    effectively from [habitual] grace, then it does so freely. Therefore, the infusion and reception of [habitual] grace can be
    understood without reference to the [dispositive] act which is elicited, if this act is elicited freely by the very
    grace, even in the instant of reception. Therefore, such a disposition does not pertain to the reception of
    [habitual] grace itself, if it \emph{freely} proceeds from the [habitual] grace already 
    received.\footnote{d. 28, a. 2, n. 9} \\

    \noindent \textbf{Second Confirmation.} Justification takes place in the sacrament of penance through
    attrition. But this attrition does not proceed effectively from grace, for then it would be contrition. Therefore,
    similarly, when justification takes place through contrition, it stands to reason that the very motion of contrition
    is not elicited by habitual grace, but from actual grace, and through another distinct action of infusing
    grace.\footnote{d. 28, a. 2, n. 10}

\section*{An Explication of the Mind of St. Thomas}

    Although St. Thomas says that the infusion of grace is first among those things which are required for
    justification, and consequently is also first with respect to the dispositions toward it, nevertheless by the notion
    of the infusion of grace, St. Thomas does not only understand the production of habitual grace, but also the motion
    of god by which he disposes us through the means of actual grace. We can gather this from the very words he uses in
    the article that we are considering:
    
    \begin{quote}
        In any natural motion, the motion of the mover itself is first. The disposition of the matter, or the movement
        of the mobile thing is second. Last comes the end, or the term ove the movement to which the motion of the mover
        is turned. Now the very motion of God the mover is the infusion of grace, as was said 
        above.\footnote{IaIIae, q. 113, a. 8, c.}
    \end{quote}

    \noindent Therefore, since St. Thomas himself explains what he understands by ``the infusion of grace,'' namely the motion of
    God the mover toward justification, this term should not be restricted to what he says concerning the infusion of
    grace precisely for the production of habitual grace. Rather, it should be understood for the motion of God which
    moves to justification---a motion that takes place through actual grace. And although God also moves to remote
    dispositions, inasmuch as the beginning of justification is taken from these remote dispositions, nevertheless, in
    the present case, the motion of God is taken just as it touches only on the ultimate dispositions, because St.
    Thomas only treats here concerning those things which are simultaneous in the same instant with justification. This
    is implied in this article, and is explicitly stated in other places.\footnote{d. 28, a. 2, n. 11; cf. in IV
        \emph{Sent.}, dist. 17, q. 1, a. 4, qun. 2; \emph{de Veritate}, q. 28, a. 7} \\

    \noindent \textbf{First Conclusion.} In the teaching of St. Thomas, habitual grace, in the instant in which it is
    infused, concurrs formally, but not effectively, for the remission of fault and for the eliciting of contrition.

    
    
\end{document}
