\documentclass{classpaper}

\usepackage{classpaper}

\title{Notes on \emph{Hypostasis} and \emph{Persona}}
\author{Br. Peter Totleben, O. P.}
\date{}

\begin{document}


\section{Maximus the Confessor}

\subsubsection{Aidan Nichols, \emph{Byzantine Gospel}}

(pp. 88-89, commenting on \emph{Epistle} 15) ``Maximus defines nature in terms of being \emph{to einai}, the sheer fact
of existing in the order of \emph{ousia}; and he understands hypostasis as the subsisting, \emph{huphestanai}, of what
exists: as Piret paraphrases, the support and maintenance of its existing. Now we know from elsewhere in this lenthy
christological epistle that Maximus had given just this point considerable attention, by way of considering the
viewpoints of two predecessors, the so-called `Neo-Chalcedonian' theologians Leontius of Byzantium and Leontius of
Jerusalem. If Piret is right, the Maximus had spotted a weakness in each of the two Leontii. For Leontius of Byzantium,
hypostasis is when someon esubsists by or according to themself, \emph{kath'heauto}, but the Byzantine Leontius had gone
on to speak of such personhood as circumscribed by its individuality and properties. If true, this would render
impossible the affirmation of particular humanity for the Chalcedonian Saviour: his human \emph{ousia} would have to b
esimply humanity at large. For Leontius of jerusalem, hypostasis is a nature with its properties. Were the Jerusalemite
Leontius correct, and hypostsis only a further specification of \emph{ousia}---a particular \emph{ousia} in fact---then
Christ's humanity woudl have to be identified with a human hypostasis---which was presumably not the meaning of
Chalcedon, and manifestly not that of Constantinople II. Maximus' response is to marry elements of these two accounts:
hypostais is self-subsistence but not in such a way that the \emph{ousia} is sundered from its properties. The human
existence of Jesus Christ \emph{subsists}: but it does so---and here Maximus' Chalcedonian and 
Second Constantinopolitan orthodoxy is saved---enhypostatised in the Logos. And as Piret insists, in Maximus' version of
the \emph{enhypostaton} idea, itself authorised by Leontius of Byzantium, enhypostasia is not a really existing
intermediary between the hypostasis and its \emph{ousia}; rather does it locate the being and subsisting in question
\emph{in} the relationship of the \emph{ousia} to its hypostasis.''
