\documentclass{article}

\usepackage{handout}

\title{Notes on the Moral Object}
\author{Br. Peter Totleben, O. P.}
\date{}

\bibliography{bibliography}

\begin{document}

\maketitle


\section{Steven J. Jensen}

\subsection{The Conclusion to \emph{Good and Evil Actions:  A Journey Through St. Thomas Aquinas}}

It is an overstatement to suggest that intention gives species to an action.  This approach shifts the focus to the
precise formality under which something is intended.  In the end, this effectively drains the species of actions of all
of their content.  The physical nature of the action becomes all but irrelevant, and the strength of the agents
imagination becomes a force by which he can redescribe his action in innocuous terms, because formailities turn out to
have a thousand subtleties and nuances.  For example, in a craniotomy, I do not intend the death of the infant, but only
the reshaping of his skull.  (This point is against Rohnheimer and Grisez-Finnis-Boyle).

What, then, is the relationship between the exterior action and intention?  The exterior action is twofold:  the
exterior action \emph{conceived} and the exterior action \emph{performed}.  The exterior action conceived specifies the
intention as its object.  The exterior action performed, in turn, takes its species from that which is intended.

How does the exterior action conceived receive its character?  Every action has a two orders:  a teleological order
which arises from its nature and a concrete order arising from the will or deliberation.  The moral species of an action
incorporates both of these orders.  The teleological order informs the concrete order; the latter must conform to the
former.  The mistake is to ignore one of these orders (usually ignoring the former, like Rohnheimer) or to blend them
together (like Long).

The concrete order of the exterior action derives from the rational deliberation of the agent (and not from nature or
intention).  With his reason, the person conceives his action as directed to this or that effect and to this or that
end.  This deliberation moves backward from an end to be achieved to the various actions that might achieve it.  (e.g. The
soldier wants to destroy the bridge, in order to do this, he will have to blow it up, in order to blow it up he will
have to\ldots).  These ``in order to'' links only exist \emph{in ratione} and not \emph{in rerum natura}.

But, when reason forges this concrete ordering of an exterior action that it is conceiving, it is bound by the nature of
things in two ways.  First, it cannot conceive the ``in order to'' links in just any way. Reason must conceive of actions that
truly do lead to the proposed goal, nor can it skip causal connections.  (Thus, for example, the soldier cannot suppose
that snapping his fingers will destroy the bridge.  If he has determined that firing his gun will ultimately result in
the destruction of the bridge, he cannot skip the fact that firing his gun will kill the child at which he aims it).

Second, once reason has forged a concrete ordering of actions, it must at least passively recognize the properties or
attributes of the action (For example, the potter can make a vessel in any shape that he wants, but once he settles on
a shape, he must recognize that this shape will not hold liquid.  Similarly, if a thief steals a chalice, he must
passively recognize its consecration.  Thus, even if he only steals the chalice for the money, it is still sacrelige
because he stole what he recognized as a sacred thing).

This passive recognition of the properties that a concrete ordering of actions has includes apprehending whether the
proposed action relates \emph{per se} to the teleological order to the end, that its, the proper order that an action
should have.  Reason perceives the clash or harmony between the two orders.  All actions have some concrete order of
bringing about some form in some material.  They also have a teleological order:  this power of the agent (will,
reproductive faculty, etc.) is ordered to introducing a certain form in the matter.  In a good action, these two orders
conform; in an evil action the concrete order turns away from the teleological order, eliminating it as a kind of
obstacle.  (For example, He who commits bestiality directs his action upon material fit to give him pleasure, but not
upon material fit for new life, which is the teleological end of the power of reproduction.)

So, how are species determined?  Human acts take their species precisely insofar as they are good or evil.  Thus any
feature of the action can give species insofar as it makes the action good or evil.  Thus circumstances specify not in
relation to intention but in relation to the teleological end of the action.  The intention is the action of the will
which directs this power on this matter to introduce this form.  Thus the intention specifies the exterior action
performed, after it itself was specified by the exterior action as conceived in the reason.

We can distinguish three cases:  (1) Intending evil under the precisely under the formality of evil.  (2) Intending what
is evil, but under the formality of some good.  (3) Intending something good under the formality of good, but where
there are forseen evil effects that fall outside the species of the action.

In case (1), we have an instance of an evil interior act of the will, which always makes the whole action evil.  But we
should note that this does not necessarily mean that the exterior action performed is \emph{in se} evil.  For example,
almsgiving out of vainglory is evil, but not inasmuch as it is almsgiving.

In order to distinguish cases (2) and (3), recognize that a human action aims to introduce some form into some matter,
in order to in some sense ``perfect'' that matter.  In case (2), even if the action is intended under some formality of
good, nevertheless reason can recognize that what the action introduces into the form is \emph{per se} harmful.  For
example, I may be trying to achieve something good in killing this person, but nevertheless, when I act on him, I am
bringing about his death.  But reason judges that this is contrary to the order to the common good which this person
ought to share in.

In the case of (3), however, the situation is different. Here, the agent introduces a form into the matter that reason
judges is suitable.  Reason forsees an evil consequence, but recognizes that it is \emph{per accidens} to introducing a
certain form into a certain matter.  If the agent does not aim at this consequence when intending the exterior action
performed, it remains \emph{praeter intentionem} and justified.  For example, in a hysterectomy performed on the
cancerous uterus of a pregnant woman, the agent aims at introducing the form of health into the mother, while
recognizing that this will mean, \emph{per accidens}, the death of the child.  Nevertheless, the agent's intention was
formed such that he did not aim at introducing an evil form into the child, and this is not incompatible with the
child's order to the common good.

This tells us what we are to make of the act of intention:  The intention provides an order, the introduction of some
form into some subject, for the species of exterior actiosn.  The precise species of the action does not depend upon the
formality of intention.  It depends upon a comparison to reason.  This order intended---this introduction of some form
into some subject---must be compared to another order, the teleological order to the human good that reason aims to
introduce, the order that the action should have.  

Note that this also means that our description of unfit matter need
not be tautological.  Theft is not taking \emph{what is wrong to take} (a tautology), but taking \emph{what belongs to
another}, that is \emph{what is reserved for the use or procurement of another}.  Reason recognizes that this is unfit
matter for taking.

The order of our human actions, then, fits within the human good.  Indeed, it largely constitutes the human good.  Our
actions are directed to some good beyond ourselves.  If we realize this order, by moving to the end according ot the
direction of reason, the we ourselves become good, sharing in the good of the end.  Indeed, we realize or institue the
good, which is not found simply in an end state but in order.  At times, however, we reject the order to the end,
seeking our own order instead.  We turn from the good and so become evil.


\section{Kevin F. Keiser}

\subsection{From \emph{The Moral Object:  A Fresh Look}}


Kevin Keiser. ``The Moral Object: A Fresh Look.'' \emph{The Thomist} 74(2010) 237-82. \\

For light in trying to understand what St. Thomas means by object, intentions, and circumstances, we should not turn
primarily to IaIIae q. 18.  Rather, we should look primarily to IaIIae, q. 20, a. 3, assuming the content in q. 18,
a. 6 only.  (``This is a better approach \emph{quad nos}'')  Such a reading would subsume the consideration of the moral
act from its ``three sources'' into a consideration of the one act that is composed of the exterior act and the interior
act.

\subsubsection{Clarification of Terms:  Object}

Object literally means ``that which is borne upon.''  It is a relational term which implies something that does the
``bearing upon.''  This is the agent. By means of an act of one of the agent's powers, the agent and object 
are united or mediated to each other in some way.  Acts and powers are correlative.  The object is the term of an active
power, while the object is the principle of a passive power.  Because a power is a principle of many actions, the
\emph{ratio} of the object of the power will have a relation of universality to the \emph{ratio} of the object of the
act or passion.

\paragraph*{The Object of the Exterior Act}

The exterior act is the act that is commanded by the will.  The will
follows the direction of the reason, and it directs either one of the powers of the soul, or one of the bodily members
to perform the action.  The object of the exterior act of the will is the \emph{materia circa quam}, that is, the
non-action upon which this exterior act bears.  When we call the object of the exterior act the \emph{materia circa
quam}, we do not necessarily mean that the object of the exterior act is a corporeal substance.  We just mean that the
exterior act bears upon something that is not identical with itself or the powers that execute it, although the
\emph{materia circa quam} must have some \emph{ratio} that contains it within the common object of these executing
powers.

There is one more complexity.  Note that the object of the exterior act, the \emph{materia circa quam}, is the object of
the power that executes the action.  But, the act of the will itself, which commands the execution, has its own object.
The object of the exterior act, then is contained under this object of the will.  That is, the object of the exterior
act, the \emph{materia circa quam} is the object of the act of the power immediately, but it is \emph{also} the object 
of the will insofar as such an object is perceived as a good to which the subject must be united by means of the action
of the bodily members and/or the powers of the soul that are imperated by the will.  More on this below.

\paragraph*{The Object of the Interior Act of the Will}

The interior act of the will is an act that is \emph{elicited} from the will, and not commanded by the will.  That is,
the interior act of the will is an act of willing as such; the object of this act is the thing willed.  The formal
object of the will is the good in general, so the object of the interior act of the will must be something proposed by
reason as good.  But the good has the character of an end.  Thus, the object of the interior act of the will is an end.
that is, for the will object and end come to the same thing.  

But, since man's action always involves a series of many
ordered ends, many of the ends of the will are themselves means to further ends.  Inasmuch as a good is seen as a means,
it has the \emph{ratio} of good only by participating in the goodness of the end to which it is ordered.  Therefore, we
can further distinguish the elicited act of the will into two different aspects, each of which has its own object:

\begin{enumerate}

    \item \textbf{The act of choice,} which bears upon that which is perceived as a good insofar as it is
        ordered to another end.  The object of the act of choice is precisely the exterior act itself in those actions
        that involve the bodily members (see below).  The object of the act of choice (i.e. the exterior act under the
        formality of the good) is what people usually mean by ``the moral object'' as distinct from the ``intention''

    \item \textbf{The act of intention,} which bears upon that which is perceived as good insofar as it is an end to be
        acquired through a means.  The object of this act is some further end beyond the immediate exterior action to be
        performed, i.e. it is an end for the sake of which the exterior action is performed.  This further end is also a
        moral object, since it is an object of the will, the principle of moral acts.

\end{enumerate}

\subsubsection{Clarification of Terms:  Intention}

The basic sense of the word intention implies a kind of pursuit.  It indicates a ``tending toward'' or ``stretching out
toward'' something.  Now, in a sense, everything that is a mixture of potency and act tends toward something.  But,
intention is especially said of cognitive beings---animals and men---who have either sense or intellectual knowledge.
This is because a cognitive faculty posesses the form of another being as other.  Actualized by this form, the cognitive
faculty can ``proceed outside of itself.''  Thus the Latin \emph{intendere} means ``to attend to, to direct oneself to,
or to apply one's vision to''  As much as our knowledge is intentional, our will is moreso.  This is because it is
conjoined to reason, which is open to a consideration of all being, and hence not determined to one.  Therefore, the
rational agent can order his own acts to a determinite end, and by the will the agent can direct himself to this end.
Also, note that intention implies some distance between the agent and the thing towards which he tends.  If the end is
immediately before someone, it is an object of choice.

There are six analogous uses of the word ``intention''

\begin{enumerate}

    \item  The application of one's cognitive powers to teh act of knowing something
    \item  The product of an act of knowing (ie. first and second intentions)
    \item  The motion of any appetite towards a good.
    \item  The motion of the will towards the good/end.  This is a very common use of ``intention,'' but should more
                properly be called the \emph{voluntas finis} or \emph{voluntas}.  It is what we mean when we say that our act is
                ``intentional,'' that we meant to do it, that we did it knowingly and willingly.  This can create some
                confusion, because in this sense, even an act of choice can be called intentional.
    \item  The motion of the will towards a good specifically insofar as it is an end to be acquired through a means.
                This is the most proper notion of ``intention,'' the one by which it is distinguished from choice.
    \item  The intended end itself.  That is, the object of the act of intention.  This is a very common use of the word
                today, especially when talking about the three sources of morality.  Here ``intention'' usualy refers to
                the end intended, rather than to the intention.

\end{enumerate}

For our purposes, (4) and (5) will be the most important.  Elizabeth Anscombe has remarked about this distinction: 

``The reason why people are confused about intention\ldots is this: They don't notice the difference between `intention'
when it means the intentionalness of the thing you're doing---that you're doing this on purpose---and when it means a
further or accompanying intention with which you do the thing.''

\subsubsection{Clarification of Terms: Circumstance}

Circumstance literally means ``that which stands around.'' When we are discussing human actions, ``circumstance''
basically means the non-defining conditions under which the act is done. Analogically, circumstance stands in relation
to act as accident to substance.

But, one act can bear diverse considerations of reason. Thus, something can be a circumstance under one consideration of
reason, and yet be a defining characteristic of the act under another consideration of reason. An important case of this
phenomenon is the distinction between the \emph{natural} species of the act, and the \emph{moral} species of the act:

\begin{enumerate}

    \item \textbf{The Natural Species.} To consider an act according to its natural species is just to consider the
        substance of the act absolutely, as it proceeds from the natural active power, along with its natural effects.
        This is all that is needed to know the natural species of the act.

    \item \textbf{The Moral Species.}  To consider an act according to its moral species is to consider the act as it
        relates to reason. Whereas the natural species considers the act as proceeding from the exterior power, the
        moral species considers the act as an object chosen and commanded by an act of the will.

\end{enumerate}

For example, consider the marital act as opposed to the adulterous act. Both of these have the same natural species
(they are the same act of the same power), but they have different moral species, because they have different relations
to reason. With respect to the natural species, it is merely circumstantial that in one act the partner is a spouse and
in the other act the partner is not. But with respect to the moral species, this is a defining characteristic of the
act.

Not only can two acts have the same natural species but different moral species; they can also have two different
natural species, but the same moral species, at least generically. For example, killing a guilty person and freeing an
innocent person are both moral species of justice.

The standard set of circumstances are:  \emph{who, what, when, where, why, how,} and \emph{by what assistance.} But this
can be a little confusing. \emph{What} and \emph{why} are important circumstances, but they also seem to answer the
question of the object and the intention, respectively. What gives?

The key is to notice that what is a circumstance depends on precisely what is being considered. So, sometimes
\emph{what} and \emph{why} yield circumstances, but sometimes they give species to the act itself. If, for instance,
\emph{why} referrs to a more ultimate end of the agent, then it is a circumstance. But, on the other hand, if \emph{why}
referrs to the immediate end of the agent, and we consider the act morally, then it is not a circumstance, but specifies
the act as its intention. Similarly, if \emph{what} referrs to the object of the exterior act and its effects, then it
is a circumstance. But, if \emph{what} referrs to the act considered morally, then it is not a circumstance, but
specifies the act itself.

Consider the action of taking money from a church's poor box. The external action, the act in its natural species, is
``collecting money.'' From this perspective, the fact that the money does not belong to the collector is circumstantial
to the act. But this act can also be considered in its moral species, as chosen and commanded by the will and as such
related to reasion, as an act of theft. From this perspective, the fact that the money does not belong to the collector
is not merely circumstantial; it gives species. The act can be considered under another moral species, that of
sacrelige. The fact that the money taken was taken from a church is circumstantial to the act when it is considered as
an act of theft, but is not circumstantial when the act is considered as an act of sacrelige. Moreover, the act must
have all of these characteristics. It is unavoidably both theft and sacrelige, because reason can recognize that the
action of ``collecting money'' is inappropriate when performed on money that is not one's own, and in fact has been
given to the Church for a sacred purpose.

So, some circumstances do not imply \emph{primo et per se} a new fittingness or repugnance to reason (taking a little bit of money that
isn't one's own vs. taking a lot of money that isn't one's own). But any circumstance that does entail a special
fittingness or repugnance \emph{primo et per se} becomes the \emph{differentia} establishing a new moral species,
another kind of willing all the way down to the \emph{species specialissima}. As such, it is no longer just a
circumstances, but an object that specifies the will.

What all of this means is that circumstances can be considered in two different ways:

\begin{enumerate}

    \item \textbf{Circumstances with respect to the natural act (ie. the exterior act) or any more generic consideration
        of the act.} 

\end{enumerate}

\end{document}
